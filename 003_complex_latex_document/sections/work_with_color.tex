\section[Работа с текстом]{
\textcolor{white}{Работа с} \label{second}
    \colorbox{ashgrey}{
        \textcolor{red}{ц}
        \textcolor{orange}{в}
        \textcolor{yellow}{е}
        \textcolor{green}{т}
        \textcolor{blue}{о}
        \textcolor{purple}{м}
        }
}
% цвет страницы
\pagecolor{mygreen}\afterpage{\nopagecolor}

% цвет текста
\textcolor{white}{This document presents several examples showing how to use the \texttt{xcolor} package to change the colour of \LaTeX{} page elements.}

% цвет списка
\begin{itemize}
\color{green}
\item First item
\item Second item
\color{orange}
\item \textcolor{orange}{\colorbox{black}{Third item}}
\item \textcolor{pink}{Fourth item}
\end{itemize}

% разделитель
\noindent
{\color{red} \rule{\linewidth}{0.5mm}}

% цвет текста
The background colour of text can also be \textcolor{red}{easily} set. For instance, you can change use an \colorbox{orange}{orange background} and then continue typing.

% фон блока
This is a sample text in black.
\textcolor{blue}{This is a sample text in blue.}
\textcolor{red}{This is a sample text in red.}
\textcolor{yellow}{This is a sample text in yellow.}
\textcolor{white}{\colorbox{orange}{This is a white sample text in orange colorbox.}}\\

% фон таблицы
\begin{tabular}{l | a | b | a | b}
\hline
\rowcolor{LightCyan}
\mc{1}{}  & \mc{1}{x} & \mc{1}{y} & \mc{1}{w} & \mc{1}{z} \\
\hline
variable 1 & a & b & c & d \\
variable 2 & a & b & c & d \\ \hline
\end{tabular}
\clearpage